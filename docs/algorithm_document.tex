\documentclass[12pt,a4paper]{article}

% 中文支持
\usepackage{ctex}

% 页面设置
\usepackage{geometry}
\geometry{left=2.5cm,right=2.5cm,top=2.5cm,bottom=2.5cm}

% 数学公式
\usepackage{amsmath}
\usepackage{amssymb}
\usepackage{amsfonts}

% 图表
\usepackage{graphicx}
\usepackage{booktabs}
\usepackage{multirow}
\usepackage{array}

% 代码高亮
\usepackage{listings}
\usepackage{xcolor}

% 算法
\usepackage{algorithm}
\usepackage{algorithmic}

% 超链接
\usepackage{hyperref}
\hypersetup{colorlinks=true,linkcolor=blue,citecolor=blue}

% 页眉页脚
\usepackage{fancyhdr}
\pagestyle{fancy}
\fancyhf{}
\fancyhead[L]{全球顶级金融风控策略研究}
\fancyhead[R]{\thepage}

% 代码样式
\lstset{
    language=Python,
    basicstyle=\ttfamily\small,
    keywordstyle=\color{blue},
    commentstyle=\color{gray},
    stringstyle=\color{orange},
    numbers=left,
    numberstyle=\tiny\color{gray},
    frame=single,
    breaklines=true,
    showstringspaces=false
}

\title{\textbf{基于集成学习的信用违约风险预测算法研究}\\[0.5em]
\large 全球顶级金融风控策略设计与实现}

\author{张峻爽}

\date{2026年2月26日}

\begin{document}

\maketitle

\begin{abstract}
本文提出了一种基于集成学习(Ensemble Learning)的信用违约风险预测模型,融合了XGBoost、LightGBM、CatBoost等梯度提升算法,并采用Top-K非负加权融合策略进行模型集成。针对金融风控领域常见的类别不平衡问题,本文采用“类权重(默认)+ 可选SMOTE(仅在交叉验证训练fold内使用)”的处理方式,以降低评估泄露与指标虚高风险。通过特征工程构建35维风险特征体系,并在评估阶段同时报告AUC、PR-AUC与KS等更贴近风控落地的指标;在决策阶段引入概率校准与成本敏感阈值优化。由于本数据集坏样本极少,本文强调以OOF指标与稳健性为第一原则,而非单一追求AUC极值。

\textbf{关键词:}信用风险;集成学习;XGBoost;LightGBM;CatBoost;TabPFN;加权融合;类别不平衡;KS;PR-AUC;概率校准;成本敏感阈值
\end{abstract}

\tableofcontents
\newpage

%==============================================================================
\section{引言}
%==============================================================================

\subsection{研究背景}

信用风险是金融机构面临的核心风险之一。随着金融科技的快速发展,机器学习技术在信用风险评估领域展现出巨大潜力。传统的信用评分方法(如FICO评分)主要依赖逻辑回归和人工经验,难以捕捉复杂的非线性关系。近年来,以XGBoost、LightGBM、CatBoost为代表的梯度提升决策树(GBDT)算法在各类数据竞赛和工业应用中取得了显著成功,成为金融风控领域的主流技术。

\subsection{研究意义}

本研究旨在构建一套全球顶级的金融风控策略体系,主要贡献包括:
\begin{enumerate}
    \item 融合多种先进集成学习算法,构建高鲁棒性的信用风险预测模型
    \item 设计全面的特征工程方案,挖掘深层次风险信号
    \item 采用类权重与(可选)fold内SMOTE处理类别不平衡问题,提升少数类识别并避免评估泄露
    \item 实现Top-K非负加权融合与“必要时回退最佳单模型”的稳健出分策略,降低小样本方差
\end{enumerate}

%==============================================================================
\section{相关工作}
%==============================================================================

\subsection{集成学习概述}

集成学习(Ensemble Learning)通过构建并结合多个学习器来完成学习任务,主要分为Bagging和Boosting两大类:

\begin{itemize}
    \item \textbf{Bagging}:通过自助采样(Bootstrap)构建多个训练集,训练多个基学习器,最终通过投票或平均进行预测。代表算法为随机森林(Random Forest)。
    \item \textbf{Boosting}:通过序列化训练,每一轮迭代关注前一轮预测错误的样本,逐步提升模型性能。代表算法包括AdaBoost、GBDT、XGBoost、LightGBM等。
\end{itemize}

\subsection{主流梯度提升算法}

\subsubsection{XGBoost}

XGBoost(eXtreme Gradient Boosting)是陈天奇提出的高效梯度提升库,其核心创新包括:
\begin{itemize}
    \item 二阶泰勒展开近似损失函数,加速收敛
    \item 加入正则化项控制模型复杂度
    \item 支持列采样和行采样,防止过拟合
    \item 高效处理缺失值
\end{itemize}

目标函数定义为:
\begin{equation}
    \mathcal{L}^{(t)} = \sum_{i=1}^{n} l(y_i, \hat{y}_i^{(t-1)} + f_t(x_i)) + \Omega(f_t)
\end{equation}

其中$\Omega(f_t) = \gamma T + \frac{1}{2}\lambda \sum_{j=1}^{T} w_j^2$为正则化项。

\subsubsection{LightGBM}

LightGBM(Light Gradient Boosting Machine)由微软研究院开发,主要特点包括:
\begin{itemize}
    \item 基于直方图的决策树算法,减少内存占用
    \item 叶子优先(Leaf-wise)的树生长策略,提高精度
    \item 支持类别特征自动处理
    \item 高效并行训练能力
\end{itemize}

\subsubsection{CatBoost}

CatBoost是Yandex开发的开源梯度提升库,针对类别特征进行了深度优化:
\begin{itemize}
    \item 采用Ordered Target Statistics处理类别特征,避免目标泄露
    \item 使用对称树(Oblivious Trees)结构,加速预测
    \item 实现Ordered Boosting减少预测偏移
\end{itemize}

\subsection{TabPFN算法}

TabPFN(Tabular Prior-Fitted Networks)是2022年由Hollmann等人提出的面向小样本表格数据的深度学习模型,其核心创新包括:

\begin{itemize}
    \item \textbf{预训练Transformer架构}:在大量合成表格数据上预训练,学习通用的表格数据表示
    \item \textbf{无需超参数调优}:基于贝叶斯神经网络,自动适应不同数据集
    \item \textbf{小样本友好}:在样本量$<1000$时表现优异,特别适合本研究的500条训练数据场景
    \item \textbf{快速推理}:单次前向传播即可完成预测,无需迭代训练
\end{itemize}

TabPFN的技术原理基于以下关键思想:
\begin{enumerate}
    \item 将表格数据视为序列,使用Transformer编码器处理
    \item 通过贝叶斯神经网络建模特征间的复杂交互
    \item 自动处理缺失值、类别特征等表格数据常见问题
\end{enumerate}

在本研究中,TabPFN作为基学习器之一参与融合策略,其AUC达到0.5857,展现了在小样本金融风控场景下的良好泛化能力。

\subsection{模型融合策略}

本文采用Top-K非负加权融合策略:
\begin{enumerate}
    \item 第一层(基学习器):训练多个不同的机器学习模型(包括XGBoost、LightGBM、CatBoost、Random Forest、Gradient Boosting、TabPFN)
    \item 融合层:按重复分层CV表现选择Top-K强模型(默认$K=3$),并将非负归一化权重用于概率融合
    \item 稳健性策略:若融合OOF AUC不优于冠军单模型,则自动回退冠军单模型
\end{enumerate}

%==============================================================================
\section{数据描述与预处理}
%==============================================================================

\subsection{数据集概况}

本研究使用的数据集包含以下文件:
\begin{itemize}
    \item \textbf{训练数据集}:500条样本,包含23个特征和1个目标变量
    \item \textbf{测试数据集}:2000条样本,用于模型预测
    \item \textbf{提交样例}:定义了预测结果的输出格式
\end{itemize}

\subsection{特征说明}

数据集包含以下类型的特征:

\begin{table}[htbp]
\centering
\caption{特征变量说明}
\begin{tabular}{lll}
\toprule
\textbf{类别} & \textbf{特征名} & \textbf{说明} \\
\midrule
\multirow{3}{*}{基本信息} 
& amount & 贷款金额 \\
& length & 贷款期限 \\
& income & 月收入 \\
\midrule
\multirow{2}{*}{类别特征}
& housing & 住房状况(租赁/自有) \\
& purpose & 贷款用途 \\
\midrule
\multirow{4}{*}{信用历史}
& overdue\_times & 逾期次数 \\
& default\_times & 违约次数 \\
& total\_default\_number & 总违约笔数 \\
& last\_overdue\_months & 最近一次逾期月数 \\
\midrule
\multirow{4}{*}{账户信息}
& account\_number & 账户数量 \\
& loan\_history & 贷款历史记录数 \\
& recent\_loan\_number & 近期贷款数量 \\
& recent\_account\_months & 近期开户月数 \\
\midrule
\multirow{4}{*}{信用卡信息}
& credict\_used\_amount & 信用卡已用额度 \\
& credict\_limit & 信用卡额度 \\
& total\_credict\_card\_number & 信用卡总数 \\
& last\_credict\_card\_months & 最近信用卡使用月数 \\
\bottomrule
\end{tabular}
\end{table}

\subsection{类别不平衡分析}

训练数据集中违约样本分布如下:
\begin{itemize}
    \item 正常样本:490条(98.0\%)
    \item 违约样本:10条(2.0\%)
\end{itemize}

类别严重不平衡。考虑到过采样在小样本场景下可能放大方差,并且若使用不当会造成评估泄露,本文采用如下策略:
\begin{itemize}
    \item \textbf{默认:类权重}(class weight / sample weight),在不改变样本分布的前提下提高少数类损失权重。
    \item \textbf{可选:SMOTE},但\textbf{仅在交叉验证每个fold的训练集内}进行过采样,验证集保持原始分布,避免指标虚高。
\end{itemize}

%==============================================================================
\section{特征工程}
%==============================================================================

\subsection{特征构建策略}

基于金融风控领域专业知识,构建了以下衍生特征:

\subsubsection{债务负担类特征}
\begin{align}
    \text{debt\_to\_income} &= \frac{\text{amount}}{\text{income} + 1} \\
    \text{total\_debt\_burden} &= \frac{\text{mortage\_number} + \text{account\_number}}{\text{income}/1000 + 1}
\end{align}

\subsubsection{信用使用类特征}
\begin{equation}
    \text{credit\_utilization} = \frac{\text{credict\_used\_amount}}{\text{credict\_limit} + 1}
\end{equation}

\subsubsection{风险评分特征}
\begin{equation}
    \text{risk\_score} = 0.3 \times \text{overdue\_times} + 0.4 \times \text{default\_times} + 0.1 \times \text{inquire\_times} + 0.2 \times (1 - \text{credit\_utilization})
\end{equation}

\subsubsection{活跃度特征}
\begin{align}
    \text{recent\_activity\_ratio} &= \frac{\text{recent\_loan\_number}}{\text{loan\_history} + 1} \\
    \text{inquiry\_frequency} &= \frac{\text{inquire\_times}}{\text{recent\_account\_months} + 1}
\end{align}

\subsection{特征总数}

经过特征工程,最终构建了\textbf{35维}特征向量,包括:
\begin{itemize}
    \item 原始特征:22个
    \item 衍生特征:11个
    \item 分箱特征:2个(income\_level, amount\_level)
\end{itemize}

%==============================================================================
\section{模型设计}
%==============================================================================

\subsection{整体架构}

本模型采用“基学习器 + Top-K加权融合 + 决策层”架构:

\begin{algorithm}[htbp]
\caption{Top-K Weighted Blend Algorithm with TabPFN}
\begin{algorithmic}[1]
\STATE \textbf{Input:} Training data $X$, labels $y$, test data $X_{test}$
\STATE \textbf{Layer 1 - Base Models:}
\FOR{each model $m \in \{XGBoost, LightGBM, CatBoost, RF, GBDT, TabPFN\}$}
    \STATE Train $m$ on $X$ with 5-fold CV
    \STATE Generate OOF (Out-of-Fold) predictions
    \STATE Generate test set predictions
\ENDFOR
\STATE \textbf{Fusion Layer:}
\STATE Select Top-K base models by repeated stratified CV performance
\STATE Compute non-negative normalized blend weights and get fused OOF predictions
\STATE \textbf{Calibration:} Fit probability calibrator on OOF predictions
\STATE \textbf{Decision:} Select threshold on calibrated OOF predictions by cost minimization
\STATE \textbf{Robustness:} If blend underperforms best single model (OOF), fallback to champion base model
\STATE \textbf{Output:} Final predictions on $X_{test}$
\end{algorithmic}
\end{algorithm}

\subsection{基学习器配置}

\subsubsection{XGBoost配置}
\begin{lstlisting}
XGBClassifier(
    n_estimators=500,
    max_depth=6,
    learning_rate=0.05,
    subsample=0.8,
    colsample_bytree=0.8,
    eval_metric='auc'
)
\end{lstlisting}

\subsubsection{LightGBM配置}
\begin{lstlisting}
LGBMClassifier(
    n_estimators=500,
    max_depth=8,
    learning_rate=0.05,
    num_leaves=31,
    subsample=0.8,
    colsample_bytree=0.8
)
\end{lstlisting}

\subsubsection{CatBoost配置}
\begin{lstlisting}
CatBoostClassifier(
    iterations=500,
    depth=6,
    learning_rate=0.05,
    loss_function='Logloss'
)
\end{lstlisting}

\subsection{SMOTE处理流程(可选)}

在本文实现中,SMOTE作为可选方案使用,并且严格限制为\textbf{仅在交叉验证每个fold的训练集内}进行过采样;验证集保持原始分布不变,以避免评估泄露。

\begin{algorithm}[htbp]
\caption{SMOTE for Class Imbalance}
\begin{algorithmic}[1]
\STATE \textbf{Input:} Imbalanced dataset $D$ with minority class $C_{min}$
\FOR{each sample $x_i \in C_{min}$}
    \STATE Find $k$ nearest neighbors of $x_i$
    \STATE Randomly select one neighbor $x_{nn}$
    \STATE Generate synthetic sample:
    \STATE $\quad x_{new} = x_i + \lambda \times (x_{nn} - x_i)$, where $\lambda \in [0, 1]$
\ENDFOR
\STATE \textbf{Output:} Balanced dataset $D'$
\end{algorithmic}
\end{algorithm}

%==============================================================================
\section{实验结果与分析}
%==============================================================================

% 自动化结果摘要(由程序运行生成,见 docs/experiment_results.tex)
% Auto-generated by src/credit_risk_model.py. DO NOT EDIT MANUALLY.
% Generated at: 2026-02-27 07:55:40

\subsection{自动化实验结果摘要(自动生成)}
训练集样本数:500,训练集坏样本率:0.0200。

\paragraph{评估与出分策略} 本项目采用OOF评估,并支持在小样本下对融合策略进行稳健性回退。

\begin{itemize}
  \item TabPFN状态:已启用(通过运行时可用性自检)
  \item 最终策略:best\_base(回退冠军单模型:catboost)
  \item 融合模型与权重:catboost=0.399; tabpfn=0.310; random\_forest=0.292
  \item 概率校准:isotonic
  \item 阈值选择方法:cost(FP成本=0.050, FN成本=1.000)
  \item 决策阈值(OOF自动选择):0.052632
  \item 最终策略OOF AUC:0.7070;OOF PR-AUC:0.0358;OOF KS:0.3408
\end{itemize}

\paragraph{基模型CV AUC(OOF泄露控制)}
\begin{table}[htbp]
\centering
\caption{基模型5折CV AUC汇总(自动生成)}
\begin{tabular}{lcc}
\toprule
模型 & AUC均值 & 95\%CI \\
\midrule
xgboost & 0.4990 & [0.3062, 0.6918] \\
lightgbm & 0.4204 & [0.1902, 0.6506] \\
catboost & 0.6102 & [0.4518, 0.7686] \\
random\_forest & 0.5806 & [0.4063, 0.7549] \\
gradient\_boosting & 0.4102 & [0.1895, 0.6309] \\
tabpfn & 0.5857 & [0.3894, 0.7820] \\
\bottomrule
\end{tabular}
\end{table}

\paragraph{测试集预测统计(自动生成)}
测试集样本数:2000,平均违约概率:0.0189,预测违约数:57,预测违约率:0.0285。

\noindent\textbf{提示:}由于坏样本极少,建议优先关注OOF PR-AUC与KS,并通过阈值策略约束拦截规模。

\noindent\textbf{生成方式:}运行\texttt{python src/credit\_risk\_model.py}将自动生成并覆盖\texttt{docs/experiment\_results.tex},以保证论文中的结果摘要与代码运行一致。

\subsection{模型性能对比}

采用5折分层交叉验证评估各模型性能,并在OOF层面输出AUC、PR-AUC与KS。由于数据规模较小且坏样本极少,指标会随数据划分产生显著波动;因此本文不固化某一次运行的数值结果,建议以程序运行日志为准,并优先关注OOF PR-AUC与KS等更贴近风控落地的衡量。

\subsection{特征重要性分析}

基于模型训练过程输出的特征重要性(以一次运行结果为例;不同数据划分/最终策略会导致重要性分数波动),Top 15特征如下:

\begin{table}[htbp]
\centering
\caption{Top 15 特征重要性排名(2026-02-26执行结果)}
\begin{tabular}{clc}
\toprule
\textbf{排名} & \textbf{特征名} & \textbf{重要性得分} \\
\midrule
1 & credit\_history\_maturity & 320.5246 \\
2 & total\_debt\_burden & 268.5148 \\
3 & income & 260.5370 \\
4 & credict\_limit & 224.0446 \\
5 & last\_credict\_card\_months & 204.0207 \\
6 & total\_balance & 202.0112 \\
7 & debt\_to\_income & 196.5077 \\
8 & credict\_used\_amount & 177.0096 \\
9 & risk\_score & 130.0068 \\
10 & credit\_utilization & 129.5109 \\
11 & amount & 129.0155 \\
12 & avg\_balance\_per\_account & 124.5073 \\
13 & total\_credict\_card\_number & 107.0226 \\
14 & loan\_history & 92.0178 \\
15 & recent\_account\_months & 71.5196 \\
\bottomrule
\end{tabular}
\end{table}

\noindent\textbf{说明:}本文更强调评估框架与稳健性策略(OOF评估、阈值选择、必要时回退最佳单模型)。特征重要性分数用于辅助理解模型关注的风险信号,不应被视为稳定不变的结论。

\subsection{关键发现}

\begin{enumerate}
    \item \textbf{信用历史最重要}:credit\_history\_maturity(信用历史成熟度)是最重要的风险指标,反映了借款人的长期信用行为模式。
    
    \item \textbf{债务负担关键}:total\_debt\_burden(总债务负担)位列第2,说明借款人的整体偿债压力是核心风险要素。
    
    \item \textbf{收入与额度影响大}:income(收入)和credict\_limit(信用卡额度)位列前茅,说明借款人的资产规模和信用额度是重要评估维度。
    
    \item \textbf{信用卡使用行为}:last\_credict\_card\_months(最近信用卡使用月数)和credit\_utilization(信用卡使用率)反映了借款人的资金紧张程度和消费行为模式。
\end{enumerate}

\subsection{预测结果统计}

测试集输出为二元标签(0/1)。预测违约数量与违约率取决于阈值策略(本文默认在OOF上进行成本最小化,同时输出坏样本率匹配与Youden参考阈值),因此应以运行日志输出为准。

%==============================================================================
\section{结论与展望}
%==============================================================================

\subsection{主要结论}

本研究构建了一套基于集成学习的信用风险预测体系,主要结论如下:

\begin{enumerate}
    \item \textbf{评估框架}:应优先采用"泄露控制"的OOF评估框架,并同时报告AUC、PR-AUC与KS,避免单一指标与评估泄露导致的过度乐观结论。
    
    \item \textbf{特征工程}:通过系统的特征工程,构建了35维风险特征体系,其中信用历史、债务负担、收入是最重要的风险指标。
    
    \item \textbf{不平衡处理}:类权重是更稳健的默认方案;若使用SMOTE,应严格限制在每个CV训练fold内以避免评估泄露。
    
    \item \textbf{模型选择}:融合策略并非在小样本下必然优于单模型;采用"融合不占优则回退最佳单模型"的策略更利于稳健落地。本次执行中,CatBoost以AUC 0.6102成为冠军模型。
    
    \item \textbf{TabPFN集成}:成功集成前沿的TabPFN模型,在小样本场景下AUC达到0.5857,排名第3,展现了深度学习在表格数据上的潜力。
\end{enumerate}

\subsection{未来展望}

\begin{enumerate}
    \item \textbf{深度学习深化}:本研究已成功集成TabPFN,未来可进一步探索TabNet、SAINT等深度表格学习模型。
    
    \item \textbf{时序特征挖掘}:利用RNN/LSTM捕捉借款人行为的时序模式,提升风险预警能力。
    
    \item \textbf{图神经网络}:构建借款人关系图谱,利用GNN挖掘关联风险和团伙欺诈。
    
    \item \textbf{模型可解释性}:引入SHAP、LIME等技术提升模型决策的透明度,满足监管要求。
    
    \item \textbf{在线学习}:实现模型的实时更新,适应市场变化和新风险模式。
    
    \item \textbf{多任务学习}:同时优化违约预测、额度授信、利率定价等多个目标。
\end{enumerate}

%==============================================================================
\section*{参考文献}
%==============================================================================

\begin{enumerate}
    \item Chen T, Guestrin C. XGBoost: A scalable tree boosting system[C]//Proceedings of the 22nd ACM SIGKDD international conference on knowledge discovery and data mining. 2016: 785-794.
    
    \item Ke G, et al. LightGBM: A highly efficient gradient boosting decision tree[C]//Advances in neural information processing systems. 2017: 3146-3154.
    
    \item Prokhorenkova L, et al. CatBoost: unbiased boosting with categorical features[C]//Advances in neural information processing systems. 2018: 6638-6648.
    
    \item Hollmann N, et al. TabPFN: A Transformer That Solves Small Tabular Classification Problems in a Second[J]. arXiv preprint arXiv:2210.14648, 2022.
    
    \item Chawla N V, et al. SMOTE: synthetic minority over-sampling technique[J]. Journal of artificial intelligence research, 2002, 16: 321-357.
    
    \item Wolpert D H. Stacked generalization[J]. Neural networks, 1992, 5(2): 241-259.
\end{enumerate}

\end{document}
